\documentclass[a4paper]{article}
\usepackage[normalem]{ulem}
\usepackage[T1]{fontenc}
\usepackage[french]{babel}
\frenchsetup{StandardLayout=true}

\newcommand{\relat}[1]{\textsc{#1}}
\newcommand{\attr}[1]{#1}
\newcommand{\prim}[1]{\uline{#1}}
\newcommand{\foreign}[1]{\#\textsl{#1}}

\title{Conversion en relationnel\\du MCD \emph{complex}}
\author{\emph{Généré par Mocodo}}

\begin{document}
\maketitle

\begin{itemize}
  \item \relat{ANIMAL} (\foreign{\prim{code espèce}}, \prim{nom}, \prim{date naissance}, \attr{sexe}, \attr{date décès}, \foreign{code espèce mère?}, \foreign{nom mère?}, \foreign{date naissance mère?}, \attr{type alimentation?}, \attr{est carnivore!}, \attr{quantité viande?}, \attr{est herbivore!}, \attr{plante préférée?})
  \begin{itemize}
    \item Le champ \emph{code espèce} fait partie de la clé primaire de la table. C'est une clé étrangère qui a migré à partir de l'entité \emph{ESPÈCE} pour renforcer l'identifiant.
    \item Les champs \emph{nom} et \emph{date naissance} font partie de la clé primaire de la table. C'étaient déjà des identifiants de l'entité \emph{ANIMAL}.
    \item Les champs \emph{sexe} et \emph{date décès} étaient déjà de simples attributs de l'entité \emph{ANIMAL}.
    \item Les champs à saisie facultative \emph{code espèce mère}, \emph{nom mère} et \emph{date naissance mère} sont des clés étrangères. Ils ont migré par l'association de dépendance fonctionnelle \emph{A MÈRE} à partir de l'entité \emph{ANIMAL} en perdant leur caractère identifiant.
    \item Un discriminateur à saisie facultative \emph{type alimentation} est ajouté pour indiquer la nature de la spécialisation. Peut être vide, du fait de l'absence de contrainte de totalité.
    \item Un champ booléen à saisie obligatoire \emph{est carnivore} est ajouté pour indiquer si on a affaire ou pas à la spécialisation de même nom.
    \item Le champ à saisie facultative \emph{quantité viande} a migré à partir de l'entité-fille \emph{CARNIVORE} (supprimée).
    \item Un champ booléen à saisie obligatoire \emph{est herbivore} est ajouté pour indiquer si on a affaire ou pas à la spécialisation de même nom.
    \item Le champ à saisie facultative \emph{plante préférée} a migré à partir de l'entité-fille \emph{HERBIVORE} (supprimée).
  \end{itemize}

  \item \relat{ESPÈCE} (\prim{code espèce}, \attr{nom latin}$^{u\_1}$, \attr{nom vernaculaire})
  \begin{itemize}
    \item Le champ \emph{code espèce} constitue la clé primaire de la table. C'était déjà un identifiant de l'entité \emph{ESPÈCE}.
    \item Le champ \emph{nom latin} était déjà un simple attribut de l'entité \emph{ESPÈCE}. Il obéit à la contrainte d'unicité 1.
    \item Le champ \emph{nom vernaculaire} était déjà un simple attribut de l'entité \emph{ESPÈCE}.
  \end{itemize}

  \item \relat{OCCUPE} (\foreign{\prim{code espèce}}, \foreign{\prim{nom}}, \foreign{\prim{date naissance}}, \prim{num. enclos}, \attr{date début!}, \attr{date fin!})
  \begin{itemize}
    \item Les champs \emph{code espèce}, \emph{nom} et \emph{date naissance} font partie de la clé primaire de la table. Ce sont des clés étrangères qui ont migré directement à partir de l'entité \emph{ANIMAL}.
    \item Le champ \emph{num. enclos} fait partie de la clé primaire de la table. Sa table d'origine (\emph{ENCLOS}) ayant été supprimée, il n'est pas considéré comme clé étrangère.
    \item Les champs à saisie obligatoire \emph{date début} et \emph{date fin} sont de simples attributs. Ils ont migré directement à partir de l'entité \emph{PÉRIODE} en perdant leur caractère identifiant. Cependant, comme la table créée à partir de cette entité a été supprimée, ils ne sont pas considérés comme clés étrangères.
  \end{itemize}

  \item \relat{PEUT COHABITER AVEC} (\foreign{\prim{code espèce}}, \foreign{\prim{code espèce commensale}}, \attr{nb. max. commensaux})
  \begin{itemize}
    \item Les champs \emph{code espèce} et \emph{code espèce commensale} constituent la clé primaire de la table. Ce sont des clés étrangères qui ont migré directement à partir de l'entité \emph{ESPÈCE}.
    \item Le champ \emph{nb. max. commensaux} était déjà un simple attribut de l'association \emph{PEUT COHABITER AVEC}.
  \end{itemize}

  \item \relat{PEUT VIVRE DANS} (\foreign{\prim{code espèce}}, \prim{num. enclos}, \attr{nb. max. congénères})
  \begin{itemize}
    \item Le champ \emph{code espèce} fait partie de la clé primaire de la table. C'est une clé étrangère qui a migré directement à partir de l'entité \emph{ESPÈCE}.
    \item Le champ \emph{num. enclos} fait partie de la clé primaire de la table. Sa table d'origine (\emph{ENCLOS}) ayant été supprimée, il n'est pas considéré comme clé étrangère.
    \item Le champ \emph{nb. max. congénères} était déjà un simple attribut de l'association \emph{PEUT VIVRE DANS}.
  \end{itemize}

\end{itemize}

\end{document}
