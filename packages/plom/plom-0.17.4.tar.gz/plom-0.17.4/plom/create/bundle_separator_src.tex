%%%%%%%%%%%%%%%%%%%%%%%%%%%%%%%%%
%% SPDX-License-Identifier: AGPL-3.0-or-later
%% Copyright (C) 2023 Andrew Rechnitzer
%% Copyright (C) 2023, 2025 Colin B. Macdonald
%% Copyright (C) 2024-2025 Forest Kobayashi
%%%%%%%%%%%%%%%%%%%%%%%%%%%%%%%%%

\documentclass[12pt,letterpaper]{article}

\usepackage[letterpaper,left=29mm,right=29mm,top=10mm,bottom=4mm]{geometry}
\usepackage{graphicx}
\usepackage{color}
\usepackage{amsmath}
\usepackage{tikz}
\usetikzlibrary{calc}
\usepackage{ifthen}


\setlength{\parindent}{0em}

%%%%%%%%%%%%%

\pagestyle{empty}


\newcommand{\thingandexample}[2]{
  {
    \def\arraystretch{.8}
    % negative space on right to align with columns not using this cmd
    \begin{tabular}[x]{@{}r@{\!\!\!}}
      #1 \\ {\footnotesize #2}
    \end{tabular}
  }
}

% QR additions
\newcommand{\placeQRcodes}[4]{
  \begin{tikzpicture}[remember picture,overlay]

    \ifthenelse{\equal{#1}{}}{}{
      \node[anchor=north east, xshift=-8mm, yshift=-8mm, line width=0.25pt, draw=black, inner sep=1pt] (qrne) at (current page.north east) {\includegraphics[angle= 90, origin=c, height=15mm]{#1}};
    }
    \ifthenelse{\equal{#2}{}}{}{
      \node[anchor=north west, xshift= 8mm, yshift=-8mm, line width=0.25pt, draw=black, inner sep=1pt] (qrnw) at (current page.north west) {\includegraphics[angle=180, origin=c, height=15mm]{#2}};
    }
    \ifthenelse{\equal{#3}{}}{}{
      \node[anchor=south west, xshift= 8mm, yshift= 8mm, line width=0.25pt, draw=black, inner sep=1pt] (qrsw) at (current page.south west) {\includegraphics[angle=-90, origin=c, height=15mm]{#3}};
    }
    \ifthenelse{\equal{#4}{}}{}{
      \node[anchor=south east, xshift=-8mm, yshift= 8mm, line width=0.25pt, draw=black, inner sep=1pt] (qrse) at (current page.south east) {\includegraphics[angle=  0, origin=c, height=15mm]{#4}};
    }
  \end{tikzpicture}
}

%%% \drawcappedbox assumes you're in a tikzpicture.
% Arguments:
% #1 -> half of width of box
% #2 -> top coordinate for box
% #3 -> bottom coordinate for box
\newcommand{\drawcappedbox}[3]{
  % For the corner caps
  \pgfmathsetmacro{\cornerlen}{.5}
  \pgfmathsetmacro{\cornerwidth}{2pt}


  \coordinate (nw) at (-#1, #2);
  \coordinate (ne) at ( #1, #2);
  \coordinate (se) at ( #1, #3);
  \coordinate (sw) at (-#1, #3);

  % Overdraw coords for making the corner ``caps''
  \coordinate (nwe) at ($(nw) + (\cornerlen, 0cm)$);
  \coordinate (nws) at ($(nw) + (0, -\cornerlen)$);

  \coordinate (swe) at ($(sw) + (\cornerlen, 0cm)$);
  \coordinate (swn) at ($(sw) + (0, \cornerlen)$);

  \coordinate (new) at ($(ne) + (-\cornerlen, 0cm)$);
  \coordinate (nes) at ($(ne) + (0, -\cornerlen)$);

  \coordinate (sew) at ($(se) + (-\cornerlen, 0cm)$);
  \coordinate (sen) at ($(se) + (0, \cornerlen)$);

  \draw[line width=\cornerwidth/2, gray!50!white, densely dotted] (nw) rectangle (se);

  \draw[line width=\cornerwidth] (nwe) -- (nw) -- (nws);
  \draw[line width=\cornerwidth] (new) -- (ne) -- (nes);
  \draw[line width=\cornerwidth] (swn) -- (sw) -- (swe);
  \draw[line width=\cornerwidth] (sew) -- (se) -- (sen);
}

\begin{document}
% In a previous version, these spacing things were necessary. I forgot
% about them until I'd finished tuning all of the spacing parameters
% above. Somebody else should probably figure out what needs to change
% to get rid of them.
\,

\vfill

% %%%%%%%%%%%%%%%%%%%%%%%%%%%%%%%%%%%%%%%%%%%%%%%%%%%%%%%%%%%%%%%%%%%%%%%%%%
\placeQRcodes{qr_crn_1}{}{qr_crn_3}{qr_crn_4}
% %%%%%%%%%%%%%%%%%%%%%%%%%%%%%%%%%%%%%%%%%%%%%%%%%%%%%%%%%%%%%%%%%%%%%%%%%%

\begin{tikzpicture}[
  remember picture,
  overlay,
  xshift=.5\linewidth, % ??? why does using \paperwidth not place it
  % correctly??
  yshift=.5\paperheight
  ]

  % Hard code letter paper ratio...someone else can adjust this later
  \pgfmathsetmacro{\hwratio}{11/8.5}
  \pgfmathsetmacro{\halfboxw}{7}
  \pgfmathsetmacro{\halfboxh}{\halfboxw * \hwratio}
  \pgfmathsetmacro{\topy}{\halfboxh - 4} % Need extra room at top
  \pgfmathsetmacro{\boty}{-\halfboxh - 0.5} % Need extra room at top


  %%%%%%%%%%%%%%%%%%%%%%%%%%%%%%%%%%%%%%%%%%%%%%%%%%%%%%%%%%%%%%%%%%%%
  % Staple alignment corner
  \node[anchor=north west, xshift=4mm, yshift=-4mm] at (current page.north west) {
    \begin{tikzpicture}
      \coordinate (a) at (0,0);
      \coordinate (b) at (2.75,0);
      \coordinate (c) at (0,-2.75);

      \filldraw[draw=black, fill=gray!30] (a)--(b)--(c)--(a);

      \clip (a)--(b)--(c)--(a);

      \node () at (-.625, .625) {\rotatebox{45}{
          \begin{tabular}{c}
             Align \\ with staples
          \end{tabular}
        }};

    \end{tikzpicture}
  };
  %%%%%%%%%%%%%%%%%%%%%%%%%%%%%%%%%%%%%%%%%%%%%%%%%%%%%%%%%%%%%%%%%%%%



  \node () at (0, \halfboxh+2.5) {\bfseries \huge Plom Bundle Separator Sheet};

  \node () at (.25, \halfboxh+1) {
    % A (brittle) self-rolled `\begin{itemize}[...] \end{itemize}'
    % equivalent that (unlike itemize) works inside tikz nodes.
    \begin{tabular}{l}
      % -2.5ex experimentally chosen for hanging indent (in case of multiline bullet point)
      \hspace*{-2.5ex}\textbullet\; Default to 20 exams per bundle \\[.5em]
      \hspace*{-2.5ex}\textbullet\; Align labeled corner with bundle exams' staples and place on top\\[.5em]
    \end{tabular}
  };

  \begin{scope}
    % These are only separately defined as a holdover from how stuff
    % was written previously. But, I think it makes it a little easier
    % to understand what's going on, so I've kept it.
    \pgfmathsetmacro{\invignotew}{7}
    \pgfmathsetmacro{\invigtopy}{\topy+3.5}
    \pgfmathsetmacro{\invigboty}{\topy-.5}

    \drawcappedbox{\invignotew}{\invigtopy}{\invigboty}

    \node () at  (0, \invigtopy-0.4) {Invigilator notes (optional)};

  \end{scope}

  \begin{scope}[yshift=-2cm]

    %% Draw in outer box
    \drawcappedbox{\halfboxw}{\topy}{\boty}

    \node () at (0, \topy-0.4) {To be filled out by scanning team (not invigilators)};

    \newcommand{\infobox}{
      \raisebox{-.6cm}{
        \begin{tikzpicture}
          \draw (0, 0) rectangle (6, 1.5);
        \end{tikzpicture}
      }
    }

    \newcommand{\checkbox}{
      \raisebox{-.1cm}{
        \begin{tikzpicture}
          \draw (0, 0) rectangle (.5, .5);
        \end{tikzpicture}
      }
    }


    \node () at (0, \topy-5.0) {
      \begin{tabular}{rl}
        \thingandexample{Exam shortcode:}{(e.g.~\texttt{24-w1-m100-mt1})}
        & \infobox \\[6ex]
        \thingandexample{Exam subunit:}{(e.g.~room or section \#)}
        & \infobox \\[6ex]
        \thingandexample{Which scanner:}{(if relevant)} &  \infobox \\[5ex]
        Tick after scanning: & \checkbox \\[2ex]
        Tick if rescanning: & \checkbox
      \end{tabular}
    };


    \pgfmathsetmacro{\noteboxw}{6}
    \draw (-\noteboxw, \boty+5.0) rectangle (\noteboxw, \boty+1);
    \node[right] () at (-\noteboxw+.1, \boty+5.0-.3) {Other scanner notes:};
  \end{scope}


\end{tikzpicture}

\clearpage

\,

\vfill
\begin{center}
  {\bfseries \huge Plom Bundle Separator (back)}
\end{center}
\vfill

% % Ensure QR codes appear on back as well
% %%%%%%%%%%%%%%%%%%%%%%%%%%%%%%%%%%%%%%%%%%%%%%%%%%%%%%%%%%%%%%%%%%%%%%%%%%
\placeQRcodes{qr_crn_5}{qr_crn_6}{qr_crn_7}{qr_crn_8}
% %%%%%%%%%%%%%%%%%%%%%%%%%%%%%%%%%%%%%%%%%%%%%%%%%%%%%%%%%%%%%%%%%%%%%%%%%%




\end{document}
